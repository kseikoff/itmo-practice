\documentclass[a4paper,14pt]{extarticle}

\usepackage[T2A]{fontenc}
\usepackage[utf8]{inputenc}
\usepackage[english, russian]{babel}

\usepackage[left=30mm, right=10mm, top=20mm, bottom=20mm]{geometry}

\usepackage{tempora}
\usepackage{setspace}
\onehalfspacing

\usepackage{titlesec}
\titleformat{\section}[block]{\bfseries\centering\MakeUppercase}{\thesection.}{1em}{}
\titleformat{\subsection}[block]{\bfseries}{\thesubsection.}{1em}{}
\titleformat{\subsubsection}[block]{\normalsize\bfseries}{\thesubsubsection.}{1em}{}

\renewcommand{\contentsname}{\hfill \textbf{СОДЕРЖАНИЕ} \hfill\null}

\usepackage{indentfirst}
\setlength{\parindent}{1.25cm}

\usepackage{amsmath, amsfonts, amssymb}
\usepackage{graphicx}
\usepackage{caption}
\usepackage{subcaption}
\usepackage{float}
\usepackage{tikz}
\usetikzlibrary{patterns}
\usepackage{cmap}
\usepackage{hyperref}
\usepackage{xcolor}
\usepackage{listings}

\definecolor{LightGray}{gray}{0.7}

\lstdefinestyle{code}{
    language=Python,
    basicstyle=\small\ttfamily,
    numbers=left,
    numberstyle=\small\color{LightGray},
    stepnumber=1,
    numbersep=5pt,
    backgroundcolor=\color{white},
    showspaces=false,
    showstringspaces=false,
    showtabs=false,
    tabsize=4,
    captionpos=b,
    breaklines=true,
    breakatwhitespace=false,
    frame=single,
    rulecolor=\color{LightGray},
    linewidth=\linewidth,
    keywordstyle=\color{blue}\bfseries,
    commentstyle=\color{green!40!black},
    stringstyle=\color{violet},
    escapeinside={\%*}{*)},
    xleftmargin=10pt,
    xrightmargin=10pt,
    framexleftmargin=0pt,
    framexrightmargin=0pt
}
\lstset{style=code}

\hypersetup{
    colorlinks=true,
    linkcolor=blue,
    filecolor=magenta,
    urlcolor=cyan,
    pdftitle={ITMO Practice},
    pdfauthor={Rumyantsev Alexey},
    pdfsubject={TCP computer-robot communication},
    pdfkeywords={LaTeX, PDF, robot, tcp},
    pdfpagemode=FullScreen,
}

\graphicspath{{src/images/}}

\begin{document}

\begin{titlepage}
    \begin{center}
        \textbf{Федеральное государственное автономное образовательное учреждение высшего образования}\\
        \textbf{«НАЦИОНАЛЬНЫЙ ИССЛЕДОВАТЕЛЬСКИЙ УНИВЕРСИТЕТ ИТМО»}\medskip\\
        \textbf{Факультет систем управления и робототехники}
        \vfill

        {\large\bfseries Отчет о}\\
        {\large\bfseries научно исследовательской работе}\medskip\\
        {\large\bfseries по теме:}\\
        {\large\bfseries «РАЗРАБОТКА АЛГОРИТМОВ ДЛЯ ВЗАИМОДЕЙСТВИЯ С РОБОТОМ-МАНИПУЛЯТОРОМ С КОМПЬЮТЕРА (С ИСПОЛЬЗОВАНИЕМ TCP)»}
        \vfill

        \begin{flushright}
            Выполнил: студент гр. R3341\\
            А. А. Румянцев\medskip\\

            Проверил: преподаватель\\
            доцент, старший научный сотрудник, инженер В. С. Громов
        \end{flushright}

        \vfill

        Санкт-Петербург\\
        2025 г.
    \end{center}
\end{titlepage}

\setcounter{page}{2}
\tableofcontents
\newpage


\section*{Введение}
\addcontentsline{toc}{section}{Введение}
\setcounter{section}{0}
В настоящее время в промышленной и других сферах все чаще
используются роботы-манипуляторы, управляемые
со специального пульта или автоматически
через загрузку программы на робота.
Более предпочтительным является вариант
управления без человека -- это безопаснее
и выгоднее. Однако роботы используют
достаточно устаревший язык программирования,
например MELFA-BASIC. Написание кода
для подобных роботов может быть неудобным,
а программы получаться громоздкими.
Разработка нового языка программирования
для роботов потребует больших вложений,
что также не выгодно. Управление с пульта,
в свою очередь,
требует от оператора высокой квалификации --
необходимы знания техники безопасности и
принципы работы оборудования. Обучение
специалиста для управления роботом-манипулятором с пульта
является ресурсоемким процессом, требующим
значительных временных и финансовых затрат.


Для повышения безопасности и эффективности
взаимодействия с роботом,
необходимо максимально отдалить человека
от робота, при этом реализовать все основные
функции для работы с ним так, чтобы их можно
было использовать из некой виртуальной централизованной
системы по нажатию кнопок. Реализовать данную идею
можно в виде программного интерфейса -- аналога физического
пульта управления роботом в
виде программы на компьютере.
Такой подход также
позволит упростить обучение специалистов для управления
роботом-манипулятором. Кроме того, программу можно
купить один раз и установить на множество компьютеров,
а разработка и покупка нескольких физических пультов управления
будет ресурсозатратным процессом.
Однако сейчас программных интерфейсов, позволяющих взаимодействовать
с роботом с компьютера сравнительно немного,
а те, что уже есть, постепенно устаревают.
Возникает необходимость написания нового
программного интерфейса для взаимодействия с роботом.
Как и любая другая программа, структурно она делится на две
части -- одна отвечает за внешний вид
и удобство управления (сам интерфейс),
другая же обеспечивает взаимодействие с роботом
на уровне, не видном пользователю.
В рамках данной работы разрабатывалась
внутрення логика программы для
взаимодействия компьютера с
роботом-манипулятором по протоколу TCP.
Пользовательская часть интерфейса при этом рассматривалась как вспомогательная.


% \begin{figure}[H]
%     \centering
%     \includegraphics[scale=0.5]{cat.jpg}
%     \caption{Кошка}
%     \label{fig:fig_name}
% \end{figure}


% \begin{figure}[htbp]
%     \centering
%     \begin{subfigure}{0.3\textwidth}
%         \includegraphics[width=\linewidth]{cat2.jpg}
%         \caption{Картинка 1}
%     \end{subfigure}
%     \hfill
%     \begin{subfigure}{0.3\textwidth}
%         \includegraphics[width=\linewidth]{cat2.jpg}
%         \caption{Картинка 2}
%     \end{subfigure}
%     \hfill
%     \begin{subfigure}{0.3\textwidth}
%         \includegraphics[width=\linewidth]{cat2.jpg}
%         \caption{Картинка 3}
%     \end{subfigure}
%     \caption{Сравнение изображений}
%     \label{fig:fig_names}
% \end{figure}

\appendix
\renewcommand{\thesection}{\Asbuk{section}}

\section{Приложение}
% \begin{lstlisting}[label=lst:cat, caption={Пример кода}]
% cats = ['Siamese cat', 'Maine Coon', 'Persian cat',
%         'Sphinx', 'Abyssinian cat', 'Scottish fold cat']

% for cat in cats:
%     print(f'Hello, this is {cat}')
% \end{lstlisting}

\end{document}